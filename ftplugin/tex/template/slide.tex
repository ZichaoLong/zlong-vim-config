% !TEX TS-program = xelatex
% !TEX encoding = UTF-8
\documentclass[xcolor=dvipsnames]{beamer}
\usepackage{ctex}
\usepackage{amsmath,amssymb,amsthm,listings,xcolor}
\usepackage{graphicx,wrapfig,multirow,diagbox,caption, subcaption,verbatim}
\usetheme{Berlin}
%使用不同的主题,其他的有:default,柏林(Berlin),哥本哈根(Copenhagen)、博阿迪利亚(Boadilla),马德里(Madrid)、新加坡(Singapore),匹兹堡(Pittsburgh),罗切斯特(Rochester),华沙(Warsaw),马尔默(Malmoe) 
% 设置文献引用格式
\usepackage{natbib}
\bibliographystyle{abbrvnat}
\setcitestyle{open={(},authoryear,close={)}}
% 设置算法环境
\usepackage{caption,algorithm,algpseudocode}
\renewcommand{\algorithmicrequire}{\textbf{input}}
\renewcommand{\algorithmicensure}{\textbf{output}}
\floatname{algorithm}{algorithm}
% 常用符号
\DeclareMathOperator*{\argmax}{argmax}
\DeclareMathOperator*{\argmin}{argmin}
\DeclareMathOperator*{\st}{s.t.}
\DeclareMathOperator*{\tr}{tr}
\newcommand{\pd}{\partial}
\newcommand{\ud}{\mathrm{d}}
\newcommand{\mr}{\mathbb{R}}
\newcommand{\ms}{\mathbb{S}}
\newcommand{\mz}{\mathbb{Z}}
\newcommand{\mn}{\mathbb{N}}
\newcommand{\mc}{\mathbb{C}}
\renewcommand\figurename{figure}
\renewcommand\tablename{table}
% 其他
\useoutertheme{infolines}
\setbeamertemplate{navigation symbols}{} % 取消导航条?
\usecolortheme[named=SkyBlue]{structure} 
\hypersetup{colorlinks=true,linkcolor=black}
\setbeamersize{text margin left=8mm, text margin right=1cm} 
% 每个小节前增加的内容.这里每节前加上目录,每小节前增加目录
\AtBeginSection[]
{
  \begin{frame}
    \tableofcontents[currentsection,hideallsubsections]
  \end{frame}
}
\AtBeginSubsection[]
{
  \begin{frame}[shrink]
    \tableofcontents[sectionstyle=show/shaded,subsectionstyle=show/shaded/hide]
  \end{frame}
}

%
\begin{document}
\title[Beamer Sample]{Beamer Sample for @THEME}
\subtitle[Beamer ver 3.06]{Based on Beamer version 3.06}
\author{龙子超}
\institute[PKU]{School of Mathematical Sciences \\ Peking University}
\date{\today}
%======封面帧
\begin{frame}[plain] 
  \titlepage
\end{frame}

\begin{frame}
  \tableofcontents%[hideallsubsections]
\end{frame}
%======演示文稿
\section{I love Li Ruonan}
\subsection{北北}
\begin{frame}{A sample slide}

  A displayed formula:

  \[
    \int_{-\infty}^\infty e^{-x^2} \, dx = \sqrt{\pi}
  \]

  An itemized list:

  \begin{itemize}
    \item itemized item 1
      \pause
    \item itemized item 2
      \pause
    \item itemized item 3
  \end{itemize}

\end{frame}
%===============
\subsection{哥哥}
\begin{frame}

  \begin{columns}[t]
    \begin{column}{0.4\textwidth}
      Here is the first column. 
      \begin{itemize}
        \item itemized item 1
        \item itemized item 2
        \item itemized item 3
      \end{itemize}
    \end{column}
    \pause
    \begin{column}{0.5\textwidth}
      I 2 Lrn.
      \[
        S=\frac{1}{\sqrt{2}}\left[
          \begin{array}{cccc}
            1&\cos\theta&\cos2\theta&\cos3\theta\\
            0&\sin\theta&\sin2\theta&\sin3\theta
        \end{array}\right]
      \]
    \end{column}
  \end{columns}

\end{frame}
\section{moi}
\begin{frame}[label=intro]{Introduction}

  \begin{definition}
    A triangle that has a right angle is called
    a \emph{right triangle}.
  \end{definition}

  \pause

  \begin{theorem}
    In a right triangle, the square of hypotenuse equals
    the sum of squares of two other sides.
  \end{theorem}

  \pause

  \begin{proof}
    We leave the proof as an exercise to our astute reader.
    We also suggest that the reader generalize the proof to
    non-Euclidean geometries.
  \end{proof}
\end{frame}
\begin{frame}{Hello,world.}
  \begin{corollary}
    Hello,world.
  \end{corollary}
  \pause
  \begin{example}
    a beamer example.
  \end{example}
\end{frame}


\begin{frame}{呼呼}{Some other slide}
  \begin{block}{}
    huhu
  \end{block}
  If you click \hyperlink{intro}{here}, you will jump to the slide
  labeled ``intro''.
  \bigskip
  \begin{block}{haha}
    Clicking \hyperlink{intro}{\beamerbutton{here}} will also
    take you to the ``intro'' slide.
  \end{block}

\end{frame}

\begin{frame}{Strategies of early stopping }

  \begin{algorithm}[H]
    \caption*{re-train: Strategy 1}
    \begin{algorithmic}
      \Require best number of training steps $i^*$
      \State set $\theta_0$ randomly;
      \State train on $X^{train},Y^{train}$ for $i^*$ steps;
    \end{algorithmic}
  \end{algorithm}
  \pause
  \begin{algorithm}[H]
    \caption*{co-train: Strategy 2}
    \begin{algorithmic}
      \Require $\theta=\theta^*$,$n$
      \State set $\epsilon=J(\theta^*,X^{subtrain},Y^{subtrain})$
      \State \textbf{while} $J(\theta,X^{validation},Y^{validation})>\epsilon$
      \State \text{\ \ }train on $X^{train},Y^{train}$ for $n$ steps;
      \State \textbf{end}
    \end{algorithmic}
  \end{algorithm}

\end{frame}

\begin{frame}[allowframebreaks]{References}

  \bibliography{ref}

\end{frame}

%====

\end{document}

